\section{Generative pre-trained language models}

\subsection{Funktionsweise}

\subsection{Anwendungsbereiche}

GPTs können für eine Vielzahl von Anwendungen eingesetzt werden, darunter:

\begin{enumerate}
    \item Textgenerierung: Erstellung von Artikeln, Blogs, Drehbüchern oder sogar Romanen.
    \item Textübersetzung: Automatische Übersetzung von Texten in verschiedene Sprachen.
    \item Textklassifizierung: Zuordnung von Texten zu Kategorien wie Stimmungen, Themen oder Genres.
    \item Frage-Antwort-Systeme: Beantwortung von Benutzerfragen auf der Grundlage von Textinhalten.
    \item Textzusammenfassung: Erstellung von kurzen Zusammenfassungen aus längeren Texten.
    \item Sentimentanalyse: Ermittlung der Stimmung oder Meinung hinter Texten.
\end{enumerate}

\subsection{Vor- und Nachteile}

Vorteile von GPTs:
\begin{enumerate}
    \item Vielseitigkeit: GPTs können für eine Vielzahl von Aufgaben und Anwendungen eingesetzt werden.
    \item Leistungsstärke: Sie sind in der Lage, menschenähnliche Texte zu generieren und komplexe Sprachmuster zu erkennen.
    \item Skalierbarkeit: GPTs können auf immer größere Modelle und Datenmengen skaliert werden, um bessere Ergebnisse zu erzielen.
    \item Transferlernen: Sie können auf einer allgemeinen Wissensbasis aufbauen und für spezifische Aufgaben angepasst werden.
\end{enumerate}

Nachteile von GPTs:
\begin{enumerate}
    \item Rechenanforderungen: Sie erfordern erhebliche Rechenressourcen und Speicherplatz für Training und Einsatz.
    \item Umweltauswirkungen: Der Energieverbrauch bei der Entwicklung und Anwendung von GPTs kann beträchtlich sein.
    \item Unvorhersehbarkeit: Die Generierung von Texten kann manchmal unerwartete oder unerwünschte Ergebnisse liefern.
    \item Ethik
\end{enumerate}