\chapter{Theoretische Grundlagen}
\label{sec:related}

Bedeutung von theoretischen Grundlagen: Basis für Verständnis, Entwicklung und Anwendung von Techniken und Methoden in der Masterarbeit
Verbindung von Schlüsselkonzepten: GPTs, Trainingsdaten und maschinelles Lernen in Bezug auf automatische Bewertung von Freitextantworten
Überblick über die Themenbereiche: GPTs, Trainingsdaten, maschinelles Lernen und Datenbasis mit verschiedenen Aufgabentypen
Ziel der Einleitung: Leser auf den folgenden Abschnitten vorbereiten, Zusammenhänge zwischen den Konzepten verdeutlichen, Relevanz der theoretischen Grundlagen für die Arbeit betonen

Theoretische Grundlagen für Verständnis von Methoden und Konzepten
2.1: Einführung in generative, vortrainierte Sprachmodelle, Funktionsweise, Anwendungsbereiche, Vor- und Nachteile
2.2: Begriff "Trainingsdaten", Bedeutung, Herausforderungen, Automatisierung der Trainingsdatenerstellung
2.3: Grundlegende Konzepte von ML-Modellen, Supervised Learning, Bewertung von ML-Modellen
2.4: Datenbasis, verschiedene Aufgabentypen für automatische Bewertungssysteme, Charakteristiken, Bedeutung für Trainingsdaten und Bewertungssysteme
Ziel: Solides theoretisches Fundament für weitere Abschnitte der Arbeit

\section{Generative pre-trained language models}

\subsection{Funktionsweise}

\subsection{Anwendungsbereiche}

\subsection{Vor- und Nachteile}

\section{Trainingsdaten}

\subsection{Definition und Bedeutung}

\subsection{Herausforderungen bei der Erstellung}

\subsection{Automatisierte Erstellung von Trainingsdaten}
\section{Maschinelles Lernen}

\subsection{Grundlagen von ML-Modellen}

\subsection{Supervised Learning}

\subsection{Bewertung von ML-Modellen}
\section{Datenbasis und Aufgabentypen}

\subsection{Beschreibung der Datenbasis}

\subsection{Aufgabentypen und ihre Charakteristiken}

\subsection{Relevanz für automatische Bewertungssysteme}