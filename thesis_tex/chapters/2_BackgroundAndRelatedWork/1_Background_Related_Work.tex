\chapter{Theoretische Grundlagen}
\label{sec:related}

Bedeutung von theoretischen Grundlagen: Basis für Verständnis, Entwicklung und Anwendung von Techniken und Methoden in der Masterarbeit
Verbindung von Schlüsselkonzepten: GPTs, Trainingsdaten und maschinelles Lernen in Bezug auf automatische Bewertung von Freitextantworten
Überblick über die Themenbereiche: GPTs, Trainingsdaten, maschinelles Lernen und Datenbasis mit verschiedenen Aufgabentypen
Ziel der Einleitung: Leser auf den folgenden Abschnitten vorbereiten, Zusammenhänge zwischen den Konzepten verdeutlichen, Relevanz der theoretischen Grundlagen für die Arbeit betonen

Theoretische Grundlagen für Verständnis von Methoden und Konzepten
2.1: Einführung in generative, vortrainierte Sprachmodelle, Funktionsweise, Anwendungsbereiche, Vor- und Nachteile
2.2: Begriff "Trainingsdaten", Bedeutung, Herausforderungen, Automatisierung der Trainingsdatenerstellung
2.3: Grundlegende Konzepte von ML-Modellen, Supervised Learning, Bewertung von ML-Modellen
2.4: Datenbasis, verschiedene Aufgabentypen für automatische Bewertungssysteme, Charakteristiken, Bedeutung für Trainingsdaten und Bewertungssysteme
Ziel: Solides theoretisches Fundament für weitere Abschnitte der Arbeit

\section{Generative pre-trained language models}

\subsection{Funktionsweise}

\subsection{Anwendungsbereiche}

GPTs können für eine Vielzahl von Anwendungen eingesetzt werden, darunter:

\begin{enumerate}
    \item Textgenerierung: Erstellung von Artikeln, Blogs, Drehbüchern oder sogar Romanen.
    \item Textübersetzung: Automatische Übersetzung von Texten in verschiedene Sprachen.
    \item Textklassifizierung: Zuordnung von Texten zu Kategorien wie Stimmungen, Themen oder Genres.
    \item Frage-Antwort-Systeme: Beantwortung von Benutzerfragen auf der Grundlage von Textinhalten.
    \item Textzusammenfassung: Erstellung von kurzen Zusammenfassungen aus längeren Texten.
    \item Sentimentanalyse: Ermittlung der Stimmung oder Meinung hinter Texten.
\end{enumerate}

\subsection{Vor- und Nachteile}

Vorteile von GPTs:
\begin{enumerate}
    \item Vielseitigkeit: GPTs können für eine Vielzahl von Aufgaben und Anwendungen eingesetzt werden.
    \item Leistungsstärke: Sie sind in der Lage, menschenähnliche Texte zu generieren und komplexe Sprachmuster zu erkennen.
    \item Skalierbarkeit: GPTs können auf immer größere Modelle und Datenmengen skaliert werden, um bessere Ergebnisse zu erzielen.
    \item Transferlernen: Sie können auf einer allgemeinen Wissensbasis aufbauen und für spezifische Aufgaben angepasst werden.
\end{enumerate}

Nachteile von GPTs:
\begin{enumerate}
    \item Rechenanforderungen: Sie erfordern erhebliche Rechenressourcen und Speicherplatz für Training und Einsatz.
    \item Umweltauswirkungen: Der Energieverbrauch bei der Entwicklung und Anwendung von GPTs kann beträchtlich sein.
    \item Unvorhersehbarkeit: Die Generierung von Texten kann manchmal unerwartete oder unerwünschte Ergebnisse liefern.
    \item Ethik
\end{enumerate}
\section{Trainingsdaten}

\subsection{Definition und Bedeutung}

\subsection{Herausforderungen bei der Erstellung}

\subsection{Automatisierte Erstellung von Trainingsdaten}
\section{Maschinelles Lernen}

\subsection{Grundlagen von ML-Modellen}

\subsection{Supervised Learning}

\subsection{Bewertung von ML-Modellen}
\section{Datenbasis und Aufgabentypen}

\subsection{Beschreibung der Datenbasis}

\subsection{Aufgabentypen und ihre Charakteristiken}

\subsection{Relevanz für automatische Bewertungssysteme}