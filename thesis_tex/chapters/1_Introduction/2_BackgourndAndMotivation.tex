\section{Hintergrund und Motivation}
\label{sec:Background}

Mögliche Punkte:
- Automatisierung in Bildungslandschaft wichtig
- Maschinelles Lernen zur Bewertung von Freitextantworten
- Verringerung von Arbeitsaufwand und objektivere Bewertung
- Herausforderung: Erstellung qualitativ hochwertiger Trainingsdaten
- Manuelle Annotation zeitaufwendig, teuer, inkonsistent
- Einsatz generativer Sprachmodelle wie ChatGPT zur Trainingsdaten-Erstellung
- Fortschritte in generativen Sprachmodellen ermöglichen menschenähnliche Texte
- Ziel: qualitativ hochwertige, vielfältige Trainingsdaten
- Verbesserung der Effizienz bei Bewertung von Freitextantworten
- Einsatzmöglichkeiten generativer Sprachmodelle in der Bildung
- Qualität der automatischen Bewertung verbessern, Zeitersparnis für Lehrkräfte