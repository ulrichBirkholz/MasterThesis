%% +++++++++++++++++++++++++++++++++
%% Setzen von scrreprt
%% +++++++++++++++++++++++++++++++++
\documentclass[
11pt,
titlepage,
a4paper,
abstracton,
twoside,
openright,
chapterprefix,
noappendixprefix,
headsepline,
footsepline,
cleardoubleplain,
bibtotoc,
liststotoc,
pointlessnumbers
]{scrreprt}

%% +++++++++++++++++++++++++++++++++
%% Einbinden von Paketen
%% +++++++++++++++++++++++++++++++++
\usepackage{istitle}
\usepackage{geometry}
\usepackage[utf8]{inputenc} 
\usepackage[T1]{fontenc}
\usepackage{ae,aecompl}
\usepackage{amsmath}
\usepackage{amsthm}
\usepackage{amscd}
\usepackage{amsfonts}
\usepackage{amssymb}
\usepackage{listings}
\usepackage{xcolor}
\usepackage{graphicx}
\usepackage{url}
\usepackage{scrlayer-scrpage}
\usepackage{bbm}
\usepackage{array}
\usepackage{booktabs}
\usepackage{threeparttable}
\usepackage{pifont}
\usepackage{placeins}
\usepackage[font=small,labelfont=bf,labelsep=colon]{caption}
\usepackage{natbib}

% Für die Verwendung deutscher Sprache
\usepackage[ngerman]{babel}
% ------------------------------------

%\bibliographystyle{abbrvnat}

% Settings for generating the PDF
% See https://www.tug.org/applications/hyperref/manual.html

\usepackage{hyperref}

\hypersetup{
	pdfinfo={ 
    		Title={Masterarbeit},
		Creator={TeX},
		Producer={pdfTeX 0.15a},
		Author={},
		CreationDate={D:20091004000000},
		ModDate={D:20130331000000},
		Subject={Masterarbeit},
		Keywords={}
	},
	pdfpagelayout=TwoColumnRight,
	pdfdisplaydoctitle=true
}



%% Settings for generating the PDF
% See https://www.tug.org/applications/hyperref/manual.html


\hypersetup{
}


 % Links klickbar, vom PDF-Viewer hervorgehoben
%% Settings for generating the PDF
% See https://www.tug.org/applications/hyperref/manual.html


\hypersetup{
	colorlinks, % colored links
	linkcolor=blue,
	filecolor=darkgreen,
	urlcolor=red,
	citecolor=green,
	hypertexnames=false,
}
 % Links klickbar, farbig hervorgehoben
% Settings for generating the PDF
% See https://www.tug.org/applications/hyperref/manual.html


\hypersetup{
	hidelinks % for hiding links altogether
}


 % Links klickbar, aber nicht hervorgehoben



%% +++++++++++++++++++++++++++++++++
%% Header
%% +++++++++++++++++++++++++++++++++

% Seitengeometrie
\geometry{a4paper,outer=38mm,inner=26mm,top=40mm,bottom=50mm}

% Definition von arg max
\DeclareMathOperator*{\argmax}{arg\,max}

% Definition von Sternchen
\newcommand{\sig}{\ding{73}}
\newcommand{\ssig}{\ding{72}}

% Definition von Häkchen und Kreuzchen
\newcommand{\h}{\ding{51}}
\newcommand{\x}{\ding{55}}

% Farben definieren
\definecolor{lightgrey}{rgb}{0.99,0.99,0.99}
\definecolor{colKeys}{rgb}{0,0,1}
\definecolor{colIdentifier}{rgb}{0,0,0}
\definecolor{colComments}{rgb}{1,0,0}
\definecolor{colString}{rgb}{0,0.5,0}

\definecolor{darkred}{rgb}{0.5,0,0}
\definecolor{darkgreen}{rgb}{0,0.5,0}
\definecolor{darkblue}{rgb}{0,0,0.5}
\definecolor{green}{rgb}{0,0.7,0}
\definecolor{blue}{rgb}{0,0,0.7}
\definecolor{red}{rgb}{0.7,0,0}
\definecolor{black}{rgb}{0,0,0}

% Quellcode
\lstloadlanguages{XML} 
\lstset{
    float=hbp,
    keywordstyle=\color{colKeys},
    stringstyle=\color{colString},
    commentstyle=\color{colComments},
    basicstyle=\texttt\small,
    identifierstyle=\color{colIdentifier},
    columns=flexible,
    tabsize=2,
    frame=single,
    extendedchars=true,
    showspaces=false,
    showstringspaces=false,
    numbers=none,
    numberstyle=\tiny,
    breaklines=true,
    backgroundcolor=\color{lightgrey},
    breakautoindent=true,
	captionpos=b,
	xleftmargin=\fboxsep,
	xrightmargin=\fboxsep,
	frameround=tttt
}

% Kopf- und Fußzeilen
\pagestyle{scrheadings}
\ihead[]{}
\chead[]{}
\ohead[]{\textsf{\headmark}}
%\ifoot[]{}
%\cfoot[]{}
%\ofoot[]{\textsf{\pagemark}}

% Absätze
\setlength{\parindent}{15pt}
%\setlength{\parskip}{2ex}

\def\topfraction{1.0} 
\def\bottomfraction{1.0} 
\def\textfraction{0.0}

\renewcommand*{\partpagestyle}{empty}



%% +++++++++++++++++++++++++++++++++
%% Start des Dokuments
%% +++++++++++++++++++++++++++++++++
\begin{document}

% Titelseite
\title{Automatische Erstellung von Trainingsdaten für die Bewertung von Freitextaufgaben mittels generative pre-trained language models}
\author{Ulrich Birkholz}
\matriculationid{2114780} 
\studyProgram{Praktische Informatik} 
\tutor{Professor~Dr.-Ing.~Torsten Zesch}
\thesistype{Masterarbeit}

\logo{figures/logo_uni.jpg}
\timeperiod{April 2023}{October 2023}
\pagenumbering{alph}
\maketitle

% Erklärung
\cleardoublepage
\pagestyle{empty}
\begin{center}
\Large{\textsf{\textbf{Erklärung}}}
\end{center}
\vspace{0.8cm}
Hiermit erkläre ich, dass ich die vorliegende Arbeit ohne fremde Hilfe selbstständig
verfasst und nur die angegebenen Quellen und Hilfsmittel benutzt habe. Ich versichere
weiterhin, dass ich diese Arbeit noch keinem anderen Prüfungsgremium vorgelegt habe.

Hagen, 17. April 2023
\\[1cm]
.................................................\\[0.2cm]
Ulrich Birkholz
\pagestyle{empty}


% Kurzzusammenfassung (Abstract)
\setcounter{page}{2}
\begin{abstract}
\thispagestyle{plain}
very abstract

\end{abstract}

% Inhaltsverzeichnis
\pagestyle{scrheadings}
\setcounter{tocdepth}{2}
\tableofcontents

% Text
\cleardoublepage
\pagenumbering{arabic}
\pagestyle{scrheadings}
\sloppy

% CAREFUL: This is just an example what the outline of your thesis might look like. Your topic might require a different approach:
% e.g. you might not have a Data collection chapter, because you use existing data. Your Experimental Studies might be so numerous that they should be split into two chapters, maybe you need a chapter for a user study. So please asapt this template according to your specific needs.

\chapter{Einleitung}
\label{chap:Introduction}

\section{Hintergrund und Motivation}
\label{sec:Background}

\section{Zielsetzung und Forschungsfragen}
\label{sec:Questions}

Mögliche Forschungsfragen:

\begin{enumerate}
\item Wie können generative pre-trained language models, wie z. B. ChatGPT, verwendet werden, um Trainingsdaten für die automatische Bewertung von Freitextaufgaben zu generieren?
\item Welche prompt engineering Techniken sind am effektivsten, um Trainingsdaten mit unterschiedlichen Charakteristiken zu erzeugen, wie beispielsweise verschiedene Antwortarten, inhaltliche Richtigkeit oder Fehler in den Antworten?
\item Inwieweit beeinflussen verschiedene Hyperparameter wie Model Temperatur oder Bearbeitungszeit, des generativen pre-trained language models die Qualität und Vielfalt der erzeugten Trainingsdaten?
\item Inwiefern kann die Qualität der automatisch generierten Trainingsdaten gewährleistet werden, und welche Methoden sind am besten geeignet, um die Qualität der generierten Daten zu überprüfen und sicherzustellen?
\item Wie performant sind maschinelle Lernmodelle, die auf Basis der automatisch generierten Trainingsdaten trainiert wurden, im Vergleich zu Modellen, die mit manuell erstellten Trainingsdaten trainiert wurden?
\item Wie wirkt sich Data Augmentation, wie beispielsweise Paraphrasierung oder Textvervollständigung, auf die Qualität und Vielfalt der generierten Trainingsdaten aus und inwiefern kann dies zur Verbesserung der maschinellen Lernmodelle beitragen?
\item In welchem Ausmaß beeinflussen die Kosten für das Training von generativen Sprachmodellen die Rentabilität des Einsatzes von ChatGPT zur automatischen Bewertung von Freitextaufgaben im Vergleich zu manuellen Bewertungsmethoden?
\item Wie lassen sich durch die Kombination von automatisch generierten Trainingsdaten und manuell erstellten Trainingsdaten die Kosten und der Zeitaufwand für die Erstellung von Trainingsdaten minimieren, ohne die Qualität der maschinellen Lernmodelle zu beeinträchtigen?
\end{enumerate}
\chapter{Theoretische Grundlagen}
\label{sec:related}

Bedeutung von theoretischen Grundlagen: Basis für Verständnis, Entwicklung und Anwendung von Techniken und Methoden in der Masterarbeit
Verbindung von Schlüsselkonzepten: GPTs, Trainingsdaten und maschinelles Lernen in Bezug auf automatische Bewertung von Freitextantworten
Überblick über die Themenbereiche: GPTs, Trainingsdaten, maschinelles Lernen und Datenbasis mit verschiedenen Aufgabentypen
Ziel der Einleitung: Leser auf den folgenden Abschnitten vorbereiten, Zusammenhänge zwischen den Konzepten verdeutlichen, Relevanz der theoretischen Grundlagen für die Arbeit betonen

Theoretische Grundlagen für Verständnis von Methoden und Konzepten
2.1: Einführung in generative, vortrainierte Sprachmodelle, Funktionsweise, Anwendungsbereiche, Vor- und Nachteile
2.2: Begriff "Trainingsdaten", Bedeutung, Herausforderungen, Automatisierung der Trainingsdatenerstellung
2.3: Grundlegende Konzepte von ML-Modellen, Supervised Learning, Bewertung von ML-Modellen
2.4: Datenbasis, verschiedene Aufgabentypen für automatische Bewertungssysteme, Charakteristiken, Bedeutung für Trainingsdaten und Bewertungssysteme
Ziel: Solides theoretisches Fundament für weitere Abschnitte der Arbeit

\section{Generative pre-trained language models}

\subsection{Funktionsweise}

\subsection{Anwendungsbereiche}

GPTs können für eine Vielzahl von Anwendungen eingesetzt werden, darunter:

\begin{enumerate}
    \item Textgenerierung: Erstellung von Artikeln, Blogs, Drehbüchern oder sogar Romanen.
    \item Textübersetzung: Automatische Übersetzung von Texten in verschiedene Sprachen.
    \item Textklassifizierung: Zuordnung von Texten zu Kategorien wie Stimmungen, Themen oder Genres.
    \item Frage-Antwort-Systeme: Beantwortung von Benutzerfragen auf der Grundlage von Textinhalten.
    \item Textzusammenfassung: Erstellung von kurzen Zusammenfassungen aus längeren Texten.
    \item Sentimentanalyse: Ermittlung der Stimmung oder Meinung hinter Texten.
\end{enumerate}

\subsection{Vor- und Nachteile}

Vorteile von GPTs:
\begin{enumerate}
    \item Vielseitigkeit: GPTs können für eine Vielzahl von Aufgaben und Anwendungen eingesetzt werden.
    \item Leistungsstärke: Sie sind in der Lage, menschenähnliche Texte zu generieren und komplexe Sprachmuster zu erkennen.
    \item Skalierbarkeit: GPTs können auf immer größere Modelle und Datenmengen skaliert werden, um bessere Ergebnisse zu erzielen.
    \item Transferlernen: Sie können auf einer allgemeinen Wissensbasis aufbauen und für spezifische Aufgaben angepasst werden.
\end{enumerate}

Nachteile von GPTs:
\begin{enumerate}
    \item Rechenanforderungen: Sie erfordern erhebliche Rechenressourcen und Speicherplatz für Training und Einsatz.
    \item Umweltauswirkungen: Der Energieverbrauch bei der Entwicklung und Anwendung von GPTs kann beträchtlich sein.
    \item Unvorhersehbarkeit: Die Generierung von Texten kann manchmal unerwartete oder unerwünschte Ergebnisse liefern.
    \item Ethik
\end{enumerate}
\section{Trainingsdaten}

\subsection{Definition und Bedeutung}

\subsection{Herausforderungen bei der Erstellung}

\subsection{Automatisierte Erstellung von Trainingsdaten}
\section{Maschinelles Lernen}

\subsection{Grundlagen von ML-Modellen}

\subsection{Supervised Learning}

\subsection{Bewertung von ML-Modellen}
\section{Datenbasis und Aufgabentypen}

\subsection{Beschreibung der Datenbasis}

\subsection{Aufgabentypen und ihre Charakteristiken}

\subsection{Relevanz für automatische Bewertungssysteme}
\chapter{Konzeption der Trainingsdaten}
\label{sec:koncept}

\section{Erstellung der Prompt}

\subsection{Anforderungen an die Prompt}

\subsection{Formulierung der Prompt}

%-> Dokumentation des Prozesses der Formulierung
%-> Verknüpft mit Model und Hyperparameter

\subsection{Wahl des Models}

%-> Tabelle Modell / Ergebnis falls interessant

\subsection{Wahl der Hyperparameter}

% -> Tabelle Parameter Wert / Ergebnis falls interessant

\subsection{Sicherstellen der Datenqualität}

%-> Auch wenn eine Hohe Robustheit Beobachtet werden sollte, keine Garantie auf Reproduzierbarkeit bei Verwendung von anderen Versionen von ChatGPT.
%-> Robustheit gegenüber verschiedener Fragestellungen
\section{Automatisierte Erstellung der Trainingsdaten}

\subsection{Datenerhebung und -verarbeitung}

\subsection{Datenaufbereitung und -bereinigung}
\chapter{Training und Testen des ML-Modells}

\section{Trainieren des ML-Modells}

\subsection{Auswahl des ML-Algorithmus}

\subsection{Trainieren des Modells}
\section{Testen des ML-Modells}

\subsection{Erstellung und Auswahl der Testdaten}

\subsection{Durchführung der Tests}
\chapter{Ergebnisse und Evaluation}
\label{sec:Experiments}

\section{Bewertung des ML-Modells}

\subsection{Performanzvergleich mit manuell erstellten Testdaten}

KPIs, Vergleich zwischen dem manuell erstellten Modell und dem Modell, das mit automatisch generierten Testdaten erstellt wurde (Wenn zeit ist könnte man auch vollständig manuell bewertetete Datensätze mit einbringen):
\begin{itemize}
  \item Genauigkeit, Präzision und Recoil der Auswertung (Am wichtigsten)
  \begin{itemize}
    \item Manuelle Auswertung aller Bewertungen (nach dem Prinzip der Doppelblindstudie)
    \item Einteilung der Qualität von 1 - 10 (mit Begründung)
    \item Definition fester Bewertungskriterien um subjektive Bewertungen zu minimieren (TBD).
  \end{itemize}
  \item F1-Score: 2 * (Precision * Recall) / (Precision + Recall)
  \item Geschwindigkeit der jeweiligen Systeme (Bewertungen / Sec)
  \item Ressourcenauslastung
\end{itemize}


\subsection{Evaluation der Ergebnisse}

\subsection{Diskussion der Ergebnisse}

\subsection{Interpretation der Ergebnisse}

\subsection{Einschränkungen und Limitationen}


\chapter{Zusammenfassung und Ausblick}
\label{sec:Conclusion}

\section{Zusammenfassung der Ergebnisse}

What was done?
\section{Fazit}

What was learnt?
\section{Ausblick}

What can/has to be/may be done in future research? Impact on other branches of science? society?


% Anhang
\cleardoublepage
\pagenumbering{roman}
\part*{Appendix}
\begin{appendix}
\include{master_thesis/struktural/Appendix}
\end{appendix}

% Verzeichnisse
\cleardoublepage
\listoffigures
\listoftables
\bibliographystyle{master_thesis/format/natdin_lat}
\bibliography{literature}



\end{document}
